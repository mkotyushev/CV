\documentclass[11pt,a4paper]{moderncv}

\moderncvtheme[purple]{classic}
\usepackage[utf8]{inputenc}
\usepackage[russian]{babel}
\usepackage[scale=0.82]{geometry}

\usepackage[unicode]{hyperref}
\definecolor{linkcolour}{rgb}{0,0.2,0.6}
\hypersetup{colorlinks,breaklinks,urlcolor=linkcolour, linkcolor=linkcolour}

\firstname{Михаил}
\familyname{Котюшев}
\address{}{Москва, Россия}
\mobile{+7 XXX XXX XX XX}
\email{XXXXXXXXXX@gmail.com}
% \homepage{https://lowlevelbits.org}
% \extrainfo{Software should be beautiful. Both inside and outside.}

\makeatletter
\renewcommand*{\bibliographyitemlabel}{\@biblabel{\arabic{enumiv}}}
\makeatother

\begin{document}
\maketitle

\section{Образование}
  \subsection{Университет}
  \cventry
    {Сен 2018 - Сейчас}
    {Компьютерные науки, магистрант}
    {\newline{}Московский физико-технический институт, Москва}
    {\newline\url{https://mipt.ru/}}{}
    {Факультет инноваций и высоких технологий, Кафедра анализа данных}
  \cventry
    {Сен 2014 - Июль 2018}
    {Физическая информатика, бакалавр (ср. балл 4.5)}
    {\newline{}Новосибирский государственный университет, Новосибирск, Россия}
    {\newline\url{https://www.nsu.ru/}}{}
    {Физический факультет, кафедра автоматизации физико-технических исследований}
  \subsection{Дополнительное образование}
  \cventry
    {Сен 2018 - Сейчас}
    {Студент трека Data Science}
    {\newline{}Школа анализа данных, Москва, Россия}
    {}{\newline\url{https://yandexdataschool.ru/}}{}

\section{Опыт работы}
\subsection{Официальное трудоустройство}
\cventry
  {Сен 2017 - Окт 2018}
  {Лаборант}
  {\newline{}Лаборатория нелинейной оптики волноводных систем, \newline{}
  Новосибирский государственный университет, Новосибирск, Россия}
  {\newline{}\url{https://research.nsu.ru/}}{}
  {Автоматизация физических экспериментов в области волоконной оптики. Исследования сенсорных приложений волоконных брэгговских решеток.}
\cventry
  {Сен 2017 - Окт 2018}
  {Младший инженер}
  {\newline{}Femtotech, Novosibirsk, Russia}
  {\newline{}\url{http://femtotech.ru/}}{}
  {Разработка ПО для встраиваемых систем в коммерческих проектах.}
\cventry
  {Янв 2015 - Сен 2017}
  {Младший лаборант}
  {\newline{}Лаборатория волоконной оптики, \newline{}
  Институт автоматики и элекрометрии СО РАН, Новосибирск, Россия}
  {\newline{}\url{https://www.iae.nsk.su/}}{}
  {Автоматизация физических экспериментов в области волоконной оптики.}
\subsection{Сторонние проекты и стажировки}
\cventry
  {Янв 2019 - Сейчас}
  {Инженер}
  {\newline{}Xumanless, Москва, Россия}
  {}{}
  {Разработка аппаратного обеспечения для технологического стартапа.}
\cventry
  {Дек 2017 - Янв 2018}
  {Стажер-разработчик C++}
  {\newline{}Инверсия Сенсор, Новосибирск, Россия}
  {\newline{}\url{https://i-sensor.ru/}}{}
  {Стажировка-проект. Приложение-обертка для SSH-клиента для удаленной конфигурации linux-сервера.}

\section{Технические навыки}
\cvline
  {Языки}{C++ (средний уровень), Python (высокий уровень)}
\cvline
  {Навыки}{Программирование микроконтроллеров, схемотехническое проектирование плат, математическая и алгоритмическая подготовка}
\cvline
  {Разное}{Linux, git, LaTeX}

\section{Опыт исследований}
  \subsection{Курсовые работы и дипломный проект}
  \cventry
    {Курсовая}
    {Измерение характеристик волоконных брэгговских решеток с различным перекрытием сердцевины волокна, записанных фемтосекундным излучением}
    {\newline\url{https://github.com/mkotyushev/CV/blob/master/texts/cw_4_sem.pdf}}{}{}{}
  \cventry
    {Курсовая}
    {Система контроля положения лазерного пучка в схеме записи волоконных брэгговских решеток методом поперечного сканирования сердцевины}
    {\newline\url{https://github.com/mkotyushev/CV/blob/master/texts/cw_6_sem.pdf}}{}{}{}
  \cventry
    {Диплом}
    {Система сбора и анализа данных для датчика изгибных деформаций на основе многосердцевинного волоконного световода}
    {\newline\url{https://github.com/mkotyushev/CV/blob/master/texts/diploma_thesis.pdf}}{}{}{}
  \subsection{Публикации}
  \cventry
    {2018}
    {Femtosecond pulse inscription of FBGs in multicore fibers and their applications}
    {\newline{}2018 International Conference Laser Optics (ICLO)}
    {\newline\url{https://doi.org/10.1109/LO.2018.8435450}}
    {}{}

  \cventry
    {2017}
    {Femtosecond core-scanning inscription of tilted fiber Bragg gratings}
    {\newline{}SPIE - The International Society for Optical Engineering}
    {\newline\url{https://doi.org/10.1117/12.2307132}}
    {}{}

\end{document}
